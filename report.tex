\documentclass{article}


\usepackage[utf8]{inputenc}
\usepackage[T1]{fontenc}
\usepackage[serbian]{babel}
\usepackage{graphicx}
\usepackage{longtable}
\usepackage{wrapfig}
\usepackage{rotating}
\usepackage[normalem]{ulem}
\usepackage{amsmath}
\usepackage{amssymb}
\usepackage{capt-of}
\usepackage{float}
\usepackage{fancyvrb}
\usepackage{fancyhdr}


\title{Analiza Malicioznih Bitcoin Transakcija}
\author{Stefan Nožinić}
\date{\today}

\begin{document}

\maketitle

\section{Uvod}
Ovaj seminarski rad opisuje metodologiju i rezultate analize malicioznih Bitcoin transakcija. Korišćeni podaci sadrže atribute transakcija kao što su vrednost, vreme, visina bloka, veličina, naknada i drugi. Cilj analize je identifikacija karakteristika koje razlikuju maliciozne transakcije od legitimnih.

\section{Metodologija}
Analiza je sprovedena pomoću Python skripti koje su korišćene za scraping, treniranje modela, i analizu podataka. Skripte su podeljene u sledeće fajlove:
\begin{itemize}
    \item \texttt{scrape.py}: Sadrži kod za prikupljanje podataka o Bitcoin transakcijama.
    \item \texttt{train.py}: Koristi se za treniranje modela za prepoznavanje malicioznih transakcija.
    \item \texttt{analyze.py}: Analizira rezultate i izračunava korelacije između atributa i verovatnoće da je transakcija maliciozna.
    \item \texttt{model.py}: Definiše strukturu modela korišćenog u analizi.
\end{itemize}

\section{Podaci}
Podaci korišćeni u analizi sadrže sledeće atribute:
\begin{itemize}
    \item \texttt{value}: Vrednost transakcije
    \item \texttt{time}: Vreme transakcije
    \item \texttt{block\_height}: Visina bloka
    \item \texttt{txid}: ID transakcije
    \item \texttt{is\_scam}: Oznaka da li je transakcija maliciozna
\end{itemize}
U datasetu je prisutno ukupno 225364 transakcija, od kojih je 112099 označeno kao maliciozne.

Podaci su preuzeti sa \texttt{bitcoin\_hacks\_2010to2013} baze i servisa \texttt{blockchair.info} i sačuvani u formatu \texttt{.csv}.

\section{Korelacija Atributa sa \texttt{is\_scam}}
Korelacije između različitih atributa i \texttt{is\_scam} su prikazane u Tabeli \ref{tab:correlation}.

\begin{table}[h!]
    \centering
    \begin{tabular}{|c|c|}
        \hline
        \textbf{Atribut} & \textbf{Korelacija sa \texttt{is\_scam}} \\
        \hline
        value & -0.081 \\
        time & -0.189 \\
        block\_height & -0.175 \\
        sequence & 0.004 \\
        ver & -0.083 \\
        vin\_sz & -0.068 \\
        vout\_sz & 0.142 \\
        size & 0.120 \\
        weight & 0.120 \\
        fee & 0.127 \\
        tx\_index & 0.100 \\
        block\_index & -0.175 \\
        \hline
    \end{tabular}
    \caption{Korelacija atributa sa \texttt{is\_scam}}
    \label{tab:correlation}
\end{table}

\section{Rezultati Analize}
Sledeći su ključni rezultati analize:
\begin{itemize}
    \item Prosečna vrednost transakcija:
    \begin{itemize}
        \item Maliciozne: 1.714e9
        \item Legitimne: 2.232e10
    \end{itemize}
    \item Prosečna naknada transakcija:
    \begin{itemize}
        \item Maliciozne: 2610942.90
        \item Legitimne: 669121.40
    \end{itemize}
    \item Adresa sa najvećom vrednošću malicioznih transakcija:
    \begin{itemize}
        \item \texttt{output\_addr}: 1B8n7yaXZdRShCP75...
        \item Vrednost: 2490935946000
        \item Vreme: 134
    \end{itemize}
\end{itemize}

\section{Treniranje modela}

Model je treniran koristeći Random Forest algoritam sa sledećim parametrima:

\begin{itemize}
    \item bootstrap: True
    \item maxDepth: 5
    \item numTrees: 100
    \item Ostali parametri su podešeni prema podrazumevanim vrednostima.
\end{itemize}

Najbolji model je postigao evaluacione rezultate:

% 2col table with Mera and Vrednost 
\begin{table}[h!]
    \centering
    \begin{tabular}{|c|c|}
        \hline
        \textbf{Mera} & \textbf{Vrednost} \\
        \hline
        Preciznost & 93.40\% \\
        AUC & 94.02\% \\
        \hline
    \end{tabular}
    \caption{Evaluacioni rezultati modela}
    \label{tab:evaluation}
\end{table}

\section{Zaključak}

Analiza pokazuje da postoje određene karakteristike koje mogu pomoći u identifikaciji malicioznih Bitcoin transakcija. Prosečna vrednost i naknada transakcija se značajno razlikuju između malicioznih i legitimnih transakcija. Trenirani model pokazuje visoku preciznost u klasifikaciji transakcija.

Dalje unapređenje modela može uključivati analizu dodatnih atributa i optimizaciju hiperparametara.


\end{document}
\endinput